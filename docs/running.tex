\chapter{Overview of how a calculation is done}

So you've come up with a cool problem (or it was assigned in class),
and you want to do an extended \huek\ calculation using \prog.  The
goal of this section is to get you familiar with the basic process for
moving from initial idea to final graphs and pictures.

The first step is to set up your input file. 
Basically, the input file contains a specification of the 
geometry of your molecule or extended system,
 the number of electrons in the system, any special information needed 
(the ranges of Walsh variables, any special parameters you may want to use,
 etc.) and any printing options that you want to set. \\[0.1in]



\noindent Here's a minimal input file example called {\tt foo.bind}


\shrinkspacing
\begin{verbatim}
; the name of the job
A silly example: a square of H atoms

; specification that this is a molecular problem
Molecular

;the geometry
geometry
4
1 H 0.0 0.0 0.0
2 H 1.0 0.0 0.0
3 H 1.0 1.0 0.0
4 H 0.0 1.0 0.0

; The number of electrons
Electrons
4

; printing options
PRINT
Overlap Population
Reduced overlap population
charge matrix
wavefunction
end_print
\end{verbatim}
\resumespacing


\noindent The contents of this file will be explained later. \\[0.1in]

\noindent To run the file execute the following command:

{\tt bind foo.bind}

This will create two output files: {\tt foo.bind.status} has some
status information; {\tt foo.bind.out} has all the results in it.

That's it!

If you had done a Walsh diagram or an average properties calculation
then there are some utility programs that need to be run to get the
data in shape to be displayed.  These will be discussed a later.

The data and results are now ready to be displayed using \viewprog\ or
your favorite plotting program.
