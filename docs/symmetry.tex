%%%%%%%%%%%%%%%%%%%%%%%%%%%%%%%%%%%%%%%%%%%%%%%%%%%%%%%%%%%%%%%%%%%%
\chapter{Symmetry analysis}

The symmetry analysis performed by \calcprog\ is relatively extensive
and flexible.  While the program doesn't find all
of the symmetry elements possessed by molecules, it does get a lot of
them. 

In order to make the symmetry analysis as flexible as possible, the
molecule can first be moved to the center of mass frame of reference. 
The 
moments of inertia are then found and the whole molecule is rotated
into the principle axis frame.  This allows molecules which are not
located exactly at the origin or aligned perfectly with the Cartesian
axes to be analyzed.  

The transformation to the principle axis frame is controlled by the
keyword {\sf Principle Axes}.  If this keyword is not specified, the
symmetry analysis will be done in the orientation specified in the
{\sf Geometry} section.

The program searches for the following
symmetry elements:

\begin{itemize}

\item {\bf an inversion center}

\item {\bf rotation axes} from C$_2$ through C$_8$ about the three
Cartesian axes.

\item {\bf improper rotation axes} from S$_3$ through S$_8$ about the
three Cartesian axes.

\item {\bf mirror planes} perpendicular to the Cartesian axes.

\end{itemize}

The elements found, their axes, and atoms which are equivalent under
each operation are printed to the output file.

The characters of the wavefunctions with respect to each operation are
determined by constructing the appropriate transformation matrix for
each operation and transforming the vector of atomic orbital
coefficients for each molecular orbital.  The result of this process
is the actual character of the wavefunction with respect to the
symmetry operation, not just a symmetric/anti-symmetric label.
It is important to realize that the results of this method of
displaying the results of symmetry analysis can give results which
are, at first, confusing for degenerate orbitals.  If you are looking
at the characters of a set of degerate orbitals and trying to compare
them to the characters given in a character table, it is very
important that you sum the characters of each of the members of the
degenerate set.


When a reaction coordinate is being followed, \calcprog\ first
generates all the geometries along the coordinate and determines the
symmetry elements which they possess.  The only symmetry elements
reported are those which are conserved along the entire distortion.
This means that you don't have to worry about moving from high to low
symmetry geometries or {\it vice versa}.  {\bf Note:} It is possible
that loss of symmetry elements will lead to problems in constructing a
Walsh diagram.  In these cases {\bf {\sf fit\_walsh}} will warn you.
If the diagram as constructed is incorrect, you can either change your
reaction coordinate to not include geometries with problematic
degeneracies or edit the {\tt .WALSH} file by hand to fix it.  This is
explained in more detail below in the section on fitting programs.

%%%%%%%%%%%%%%%%%%%%%%%%%%%%%%%%%%%%%%%%%%%%%%%%%%%
