%%%%%%%%%%%%%%%%%%
%%%%%%%%%%%%%%%%%%
\chapter{Other Utility Programs}

There are a number of other utilities distributed with \prog.  These
are described below.

\section{sub\_dos and add\_dos}

These are used to manipulate .DOS files.  {\tt sub\_dos} is used to
subtract two DOS curves from each other.  This is the basic operation
needed for the Crystal Orbital Displacement (COD) analysis developed
by Eliseo Ruiz and Santiago Alvarez \cite{cod}. COD is a very sensitive tool for tracking complicated interactions in the solid state.  Running {\tt sub\_dos} without any arguments will give you the correct ordering of arguments.

{\tt add\_dos} is like {\tt sub\_dos} except that it adds two DOS
curves together.

{\bf Note:} It is very important that the DOS curves used for {\tt
sub\_dos} are fit (using {\tt fit\_dos}) within the same energy window
and with the same broadening and energy step.  The programs will warn
you about this.

\section{cooperate}

{\tt cooperate} reads in the .DMAT file generated when \calcprog\ is
given the keyword {\sf Dump Distance Matrix} and generates a COOP
specification that can be pasted into an input file for \calcprog.
The output from {\tt cooperate} needs very little modification before
incorporation into an input file.  The necessary modifications are
fairly obvious.  Once again, running the program without any arguments
will give you a complete list of possible arguments.


