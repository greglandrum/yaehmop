%%%%%%%%%%%%%%%%%%
%%%%%%%%%%%%%%%%%%
\chapter{The Fitting Programs}

In order to generate nice looking DOS and COOP curves, it is necessary
to either use hundreds of k points in the calculation or to smooth the
data which is 
generated by \calcprog.  For obvious reasons, it is far more common
to adopt the latter approach.

Smoothing of DOS and COOP curves is done by putting a gaussian on each
data point, then summing up the contributions from each of the
gaussians between the data points.  This process gives rise to the
type of curves we are used to seeing.

The parts of \prog\ which perform this smoothing operation are called
{\tt fit\_dos} and {\tt fit\_coop}.  These both take the name of the
input file which was given to \calcprog\ as an argument.

Here is a sample session:

\shrinkspacing

\begin{verbatim}

% bind H_mesh.bind
% fit_dos H_mesh.bind
Enter E min: -30.0
Enter E max: 30.0
Enter broadening: 10.0
Enter Energy Step: 0.5

\end{verbatim}

\resumespacing

The broadening parameter given to the fitting programs is the exponent
of the normalized Gaussian smoothing function.  A larger broadening
parameter gives rise to sharper lines in the DOS/COOP curves.

After this smoothing process, which produces either a {\tt .DOS} or
{\tt .COOP} file, the data is ready for viewing with \viewprog.

In order to view a Walsh diagram, the program {\tt fit\_walsh} must be
run.  {\tt fit\_walsh} is run the same way as {\tt fit\_dos} or {\tt
fit\_coop}: you give it the name of the input file which was given to
\calcprog.  If you lose symmetry elements along the distortion
coordinate and degeneracies are broken, it is possible that {\tt
fit\_walsh} will get confused and generate a silly looking Walsh
diagram.  {\tt fit\_walsh} will warn you if this happens.  If the
output looks wrong in \viewprog\, you can either manually edit the {\tt 
.WALSH} file created by {\tt fit\_walsh} or rerun the calculation with
more points along the distortion and use the program {\tt dumb\_walsh},
which ignores symmetry operations.  If you take the {\tt dumb\_walsh}
route, we recommend you use at least 30--40 points along the
distortion.  Hopefully a future version of the program will have a
smarter version of {\tt fit\_walsh} so that these contortions are no
longer necessary.

