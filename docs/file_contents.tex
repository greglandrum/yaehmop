\chapter{Contents of Files}

The various programs in \prog{} produce different output files, the
names of the files may be confusing.

For a run on a file named {\tt example}, here are the names of
the files produced and the contents of those files:

\begin{itemize}

\item {\tt example.status}: status information about the job

\item {\tt example.out}: the main output file. Contains energies,
occupations, average properties, etc.

\item {\tt example.walsh}: the values of variables which were printed
out along each step of a reaction coordinate.

\item {\tt example.band}: the information needed by \viewprog{} for
constructing a band diagram.

\item {\tt example.DOS}: (generated by {\tt fit\_dos}) the information needed by \viewprog{} to
generate DOS curves.

\item {\tt example.COOP}: (generated by {\tt fit\_coop}) the
information needed by \viewprog{} to generate COOP curves.

\item {\tt example.WALSH}: (generated by {\tt fit\_walsh}) the
information needed by \viewprog{} to generate Walsh diagrams.

\item {\tt example.FMO}: contains the information needed by
\viewprog{} to generate FMO diagrams. 

\item {\tt example.MO}: contains the information needed by
\viewprog{} to generate MO pictures.

\item {\tt example.DMAT}: generated when the {\sf dump distance
matrix} keyword is used, contains the information needed by
{\tt cooperate} to automatically generate COOP specifications for
crystals. 


\end{itemize}
