\chapter{The input file}

Like many other programs, the input file for \calcprog\ is based on
keywords.  This allows the file to be broken into logical 
blocks and makes the format for constructing the file a little
less rigid.  Each keyword is described below.  Note that the input
routine (the parser) is
not case sensitive when dealing with keywords, i.e. Electrons,
electrons, and ELECTRONS will all work.

Any line in the input file which begins with a semicolon is ignored.
This allows comments to be put into the input file. It also makes it
easy to temporarily change the contents of a file, just put a
semicolon in front of anything you don't want the program to read.
Blank lines and spaces in the input are also ignored. \\[0.1in]

\noindent The first non-empty line should contain the title of the job.


\section{Keywords}
The remainder of the file contains the keywords which control the job.
These keywords are all described below.

The notation used is as follows:  

\begin{itemize}

\item Words in sans-serif style, such as {\sf foo,} are keywords.

\item Words in slanted type, such as \pvar{bar}, are
variable names used within the program.  They are used for convenience.

\end{itemize}


%%%%%%%%%
\subsection{{\sf Geometry} (required)}

The line following this keyword should have the number of atoms:
\pvar{num\_atoms}. 

If the line containing the keyword {\sf Geometry} also contains the
string {\sf Z Matrix} then the input is take to be in Gaussian style Z
matrix format.  If the line containing {\sf Geometry} also contains
the string {\sf Crystallographic} then the input is taken to be in
fractional coordinates and the {\sf Crystal Spec} keyword must be
specified.  Otherwise the input is assumed to be in Cartesian
coordinates.  If you are unfamiliar with Z matrix input, it is
described in a later chapter of this document.

The following \pvar{num\_atoms} lines should contain the atomic
coordinates and types in the following fashion:

\begin{itemize}

\item Z-matrix Input:  \pvar{number} \pvar{atom\_label} \pvar{ref1}
\pvar{r} \pvar{ref2}  \pvar{alpha} \pvar{ref3} \pvar{beta}

\item Cartesian Input: \pvar{number} \pvar{atom\_label} \pvar{X}
\pvar{Y} \pvar{Z}

\item Crystallographic Input: \pvar{number} \pvar{atom\_label} \pvar{X}
\pvar{Y} \pvar{Z}

\end{itemize}

\pvar{number} is the number of the atom.  There is no reason why the
atoms in the list have to be in increasing numeric order.

\pvar{atom\_label} is the label for the atom, it should be one or two
characters long.  If the atom label is a single asterix ({\tt *}),
then the atom is taken to be a custom type and a symbol and
parameters should be provided for it in the {\sf Parameters} section.
If the atom label is an ampersand ({\tt \&}), then the atom is taken
to be a dummy atom.

The other variables are the coordinates for the atom in whatever
system is being used.  If a reaction coordinate is being traced out
(see keyword {\sf Walsh} below), any coordinate which is an integer
multiple of 1000 is assumed to be variable.  For example: placing 2000
as the \pvar{X} coordinate of some atom would cause that coordinate to
take on the values of the second Walsh variable.


%%%%%%%%
%XXX  new XXX
%\subsection{{\sf Geom frag} (optional)}

%This keyword is used to specify a geometry fragment: a group of atoms
%that will be inserted as a single unit into the geometry of the unit
%cell.  

%%%%%%%%
\subsection{{\sf Lattice} (mandatory for extended systems)}

{\bf NOTE:} If this keyword appears in the file it {\em must} follow
the {\sf Geometry} keyword in the input file.

These are the lattice parameters.
The keyword should be followed by a line containing the dimensionality
of the crystal.  The next line contains the number of overlaps
considered along each lattice direction. The following lines should
contain the lattice vectors in the form:  \pvar{atom1} \pvar{atom2},
where \pvar{atom1} is the beginning of the vector (it is inside the
unit cell) and \pvar{atom2} is the end of the vector (it is outside
the unit cell).

{\bf NOTE:}  The ends of the lattice vectors must be the highest
numbered atoms in the geometry specification. For example, if there are
are 6 atoms defined for a 3 dimensional unit cell, then the ends of
the three lattice vectors must be numbers 4, 5, and 6.

Only a number of lattice vectors equal to the dimensionality of the
crystal need to be provided.  If you feel like putting in zeroes for
the other lattice vectors, go ahead... we can't stop you.



%%%%%%%%
\subsection{{\sf Crystal Spec} (required for use of crystallographic
coordinates)} 

This section has the stuff needed to define the crystal lattice so
that crystallographic coordinates can be used.

The first line following the {\sf Crystal Spec} keyword should contain
the lengths of each of the lattice vectors.  The next line should
contain the crystallographic angles $\alpha, \beta$ and $\gamma$.

Variables in the {\sf Crystal Spec} section can be used as variables
in Walsh diagrams in exactly the same way as variables in the {\sf
Geometry} section, i.e. by using integer multiples of 1000 as values.

For example, the following {\sf Geometry}, {\sf Lattice} and {\sf
Crystal Spec} 
sections define a body centered lattice of H atoms where the length of
the c lattice vector is the first Walsh Variable.

\clearpage

\shrinkspacing
\begin{verbatim}

Geometry Crystallographic
5
1 H 0 0 0
2 H 0.5 0.5 0.5
3 H 1 0 0
4 H 0 1 0
5 H 0 0 1

Lattice
3
5 5 5 
1 3
1 4
1 5

Crystal Spec
; a  b  c
  1  1  1000
; alpha  beta  gamma
   90     90     90

\end{verbatim}
\resumespacing


%%%%%%%%
\subsection{{\sf Electrons} (potentially required)}

The line following this keyword should have the number of valence
electrons in the molecule (or unit cell for an extended system).

%%%%%%%%
\subsection{{\sf Charge} (potentially required)}

The line following this keyword should have the charge on the molecule
(or unit cell for an extended system). \\[0.1in]

\noindent {\bf Either the {\sf Charge} or {\sf Electrons} keywords must appear in the
input file.}

%%%%%%%%
\subsection{{\sf Alternate Occups} }

This keyword is for looking at the effects of changing the number of
electrons in the unit cell upon the position of the Fermi level, the
average energy, and orbital occupations.  This is a far more efficient
way of probing these changes than rerunning the calculation with
alternate electron numbers.

On the line following the keyword, the number of alternate occupations
(\pvar{num\_occups}) should be given.  The next line contains the step
that is to be taken between occupations.

For example, the following input fragment would result in the program
doing a calculation with 5 electrons, then printing out the Fermi
level, average energy, net charges, and orbital occupations for
4.8,4.6,4.4,4.2, and 4.0 electrons per unit cell.

\shrinkspacing
\begin{verbatim}

Electrons
5

Alternate Occup
; num_occups
5
; the step
-.2
\end{verbatim}
\resumespacing



%%%%%%%%
\subsection{{\sf Parameters} (optional)}

{\bf NOTE:} If this keyword appears in the file it {\em must} follow
the {\sf Geometry} keyword in the input file.

There should be a line following this keyword for each type of custom
atom which is being defined.  Recall that a custom atom is defined by
replacing the first occurance of that atom's label in the Geometry
specification with an asterix: {\tt *}. If there are multiple custom
atom types, then define them in this section in the order in which
they occurred in the Geometry section. \\

\noindent The format of a parameter specification is:

{\small \pvar{Symbol} \pvar{Atomic\_Number}
\pvar{Num\_Valence\_Electrons} 
\pvar{n$_s$} \pvar{$\zeta_s$} \pvar{IP$_s$}
\pvar{n$_p$} \pvar{$\zeta_p$} \pvar{IP$_p$}
\pvar{n$_d$} \pvar{$\zeta1_d$} \pvar{IP$_d$} \pvar{c1}
\pvar{$\zeta2_d$} \pvar{c2}} \\

\noindent and if you're dealing with f-elements (which, like d orbitals, are described by a double zeta expansion) add:
{\small \pvar{n$_f$} \pvar{$\zeta1_f$} \pvar{IP$_f$} \pvar{c1}
\pvar{$\zeta2_f$} \pvar{c2}}
to the end of the parameter specification. \\

\noindent In this specification, the $\zeta$ values are the radial exponents of
the Slater type orbitals and the H$_{ii}$ values are the valence state
ionization potentials (diagonal elements of the hamiltonian) for each AO.
Here's an example of a section of input file where 3 different custom
atoms are defined:

\shrinkspacing
\begin{verbatim}

Geometry
8
1 *     .000000000     .000000000     .000000000
2 O    1.952000000    1.952000000     .000000000
3 *    1.952000000     .000000000    -.3
4 O    1.952000000     .000000000    1.4862
5 *    0.000000000    1.952000022    1.9422
6 O    3.904000044     .000000000     .000000000
7 O     .000000000    3.904000044     .000000000
8 O     .000000000    0.0                4.152

Parameters
O  8 6 2 2.275 -32.3 2 2.275 -14.8
Ti 22 4 4 1.075 -8.970 4 1.075 -5.400 3 4.55 -10.81 .4206 1.400 .7839
Pb 82 4 6 2.50  -16.70 6 2.06 -8.000

\end{verbatim}
\resumespacing
 
In this example, atom 1 is an O, atom 3 is a Ti, and atom 5 is a Pb.
Atoms 2,4,6,7, and 8 will use the same parameters as atom 1.  {\bf
Clarification:} The parameters specified here for O are the default
parameters. 


%%%%%%%%
\subsection{{\sf Molecular} (mandatory for molecular calculations)}

Indicates that a molecular calculation (not an extended one) is being
performed.  Otherwise it is assumed that the calculation is on an
extended system.

%%%%%%%%
%\subsection{{\sf Thin} (optional)}

%Do a THIN mode extended calculation.  This is not yet implemented, it
%will be added later to allow calculations to be run on machines with
%little available memory.

%%%%%%%%
%\subsection{{\sf Fat} (optional)}

%Do a FAT mode extended calculation.  This is the default for the
%program.

%%%%%%%%
\subsection{{\sf Just Geom} (optional)}

Do not actually do a calculation, just generate the molecular
geometry.  This is useful to check whether or not an input file is
okay before running a calculation.  It will also print the estimated
memory requirements of \calcprog\ for this run into the status file.


%%%%%%%%
\subsection{{\sf Walsh} (optional)}

Do a series of calculations along a reaction coordinate. 

The next line should contain the number of variables (reaction
coordinates): \pvar{num\_Walsh\_var}, and the number of steps to be taken
along each coordinate: \pvar{num\_steps}.  There should then be
\pvar{num\_Walsh\_var} lines consisting of a list of comma separated
\pvar{num\_steps} 
values.  

To generate the values of a variable automagically, place an
exclamation point ({\tt !}) at the beginning of the line for that
variable followed by the starting and ending values of the variable,
separated by a comma.
The program will generate \pvar{num\_steps} values between those
values.  

For example, the following sample section will generate a
reaction with coordinate with 5 steps and 2 variables.  The values of
the first variable will be automatically generated between 1.0 and
2.0, the values of the second variable are specified explicitly.

\shrinkspacing
\begin{verbatim}

Walsh
; the number of variables and number of steps
2 5
; use Auto-Walsh for the first variable:
! 1.0,2.0
100.0,100.1,100.2,100.3,100.4

\end{verbatim}
\resumespacing

Please note that, while there are two Walsh variables, they are varied
simultaneously.  There is not, at this point, a capability to vary the
Walsh variables independently in order to automatically generate a
multi-dimensional potential energy surface.

%%%%%%%%
\subsection{{\sf Symmetry} (optional)}

Find and report all the symmetry elements possessed by the molecule.
The 
characters of all wave functions with respect to these operations will
also be reported. 

{\bf Note:} As is explained in the section of this manual on symmetry
elements, {\calcprog} does not actually find {\em all} symmetry
elements.  It only finds those which are aligned with the Cartesian
axes. You can increase the number of symmetry elements which the
program finds by making sure that it align with the axes in a
reasonable manner.

%%%%%%%%
\subsection{{\sf Symm Tol} (optional)}

Allows the user to adjust the value of the tolerance used
for determining whether or not symmetry elements are present
in the molecule.  

The next line should contain the new value of \pvar{symm\_tol}.

If the position of an atom after a symmetry operation differs from
that of an atom before the symmetry operation is applied by less than 
\pvar{symm\_tol} \AA, then the two atoms are considered to be equivalent
under the symmetry operation.

%%%%%%%%
\subsection{{\sf Principle Axes} (optional)}

Determine the center of mass and principle axes of the molecule and
transform the atoms into the principle axis frame.  This can allow the
program to find more symmetry elements.  

{\bf Note:} Principle axes are only found if the {\tt meschach} library
was linked with \calcprog.  If you do not have {\tt meschach}, then
the program will just translate the molecule into the center of mass
frame before finding symmetry elements.

%%%%%%%%
\subsection{{\sf Zeta} (optional)}

Toggles self consistent variation of radial exponents.  This is under
development and if you don't know what it means, you probably
shouldn't be using it.

%%%%%%%%
\subsection{{\sf Nonweighted} (optional)}

Use the non-weighted H$_{ij}$ form \cite{wolfsberg}. 
By default the weighted H$_{ij}$
form is used in order to reduce problems arising from
counter--intuitive orbital mixing (gasp!) \cite{com1,com2}.

%%%%%%%%
\subsection{{\sf The Constant} (optional)}

Allows replacement of the value \pvar{K} used in evaluating the H$_{ij}$
elements.  The next line should contain the new value for \pvar{K}.
By default \pvar{K}=1.75.

%%%%%%%%
\subsection{{\sf Zero Overlap} (optional)}

Set some elements of the overlap matrix to zero.  The next line should
contain the number of different types of overlap being set to zero
(\pvar{num\_to\_zero}).  The next \pvar{num\_to\_zero} lines should
consist of:

\pvar{type} \pvar{contrib1} \pvar{contrib2} \pvar{which}

\pvar{type} should be either {\tt Atom} or {\tt Orbital} to indicate
which type of overlap is being zeroed.  If \pvar{which} is set to {\sf
Intercell}, then overlaps between \pvar{contrib1} and \pvar{contrib2} 
{\em between} cells will be zeroed.  If \pvar{which} is set to {\sf
Intracell} then the overlaps {\em inside} the cell will be zeroed.

{\bf NOTE:}  This option should be used with caution, as it can result
in a non-positive-definite overlap matrix and non-physical results.

%%%%%%%%
\subsection{{\sf Nearest Neighbor Contact} (optional)}

Determines the longest contact between atoms in nearest neighbor cells
that 
will be reported in the output file.  The next line should contain the
new value.  The default is 2.5 \AA. \\

\noindent {\bf NOTE:}  This keyword only makes sense for extended systems.

%%%%%%%%
\subsection{{\sf K Points} (mandatory for Average Properties
calculations on extended systems)}

A K point set for an average properties calculation.  If you want to
do a band structure, we recommend that you use the {\sf Band} keyword.

The keyword is followed by a line containing the number of K points:
\pvar{num\_KPOINTS}.   The next \pvar{num\_KPOINTS} lines should contain
the coordinates and weights of the K points themselves in the
following form:

\pvar{a} \pvar{b} \pvar{c} \pvar{weight} \\

\noindent Each K point should go on its own line.

%%%%%%%%
\subsection{{\sf Band} (optional)}

Generate a band structure.

The line following the keyword should contain the number of K points
to use along each symmetry line: \pvar{points\_per\_line}.

The next line should have the number of special points to be used:
\pvar{num\_special\_points}. 

The following \pvar{num\_special\_points} lines should have the names
and locations of the special points in the form:

\pvar{label} \pvar{x} \pvar{y} \pvar{z}

The program will generate symmetry lines connecting the special points
in the order in which they are defined.  \pvar{points\_per\_line} K
points will be generated automatically along each of these symmetry
lines.  The program will produce an additional output file containing
the information needed by \viewprog\ to plot the band structure.  If
your input file is called {\tt foo.bind}, then the band file will be
called {\tt foo.bind.band}.

For example, the following segment will generate a band diagram with
symmetry lines containing 40 K points running from $\Gamma$ to X to M and
then back to $\Gamma$:

\shrinkspacing
\begin{verbatim}

Band
; the number of points along each line
40
; the number of special points
4
Gamma 0.0 0.0 0.0
X     0.5 0.0 0.0
M     0.5 0.5 0.0
Gamma 0.0 0.0 0.0

\end{verbatim}
\resumespacing

%%%%%%%%
\subsection{{\sf FMO} (optional)}

Perform Fragment Molecular Orbital analysis.

The next line should contain the number of fragments
\pvar{num\_FMO\_frags}.  Note that \pvar{num\_FMO\_frags} should be
$\geq 1$.  There is no upper limit on the number of fragments.

The next line consists of a comma delimited list of the number of
electrons in each fragment.

The following \pvar{num\_FMO\_frags} lines consist of comma delimited
lists of the numbers of the atoms in each fragment.

In the specification of atoms for each fragment, you can use a hyphen
(dash) to indicate groups of sequentially numbered atoms.
For example, the following segment will generate 2 fragments, the
first fragment containing atoms 1, 2, 3, 4, 5 and 8 and the second fragment
containing atoms 6 and 7. \\


\shrinkspacing
\begin{verbatim}

FMO
; the number of fragments
2
; the number of electrons in each fragment
12,2
; the lists of atoms contained in each fragment
1-5,8
6-7

\end{verbatim}
\resumespacing

If a molecular calculation is being done, the data necessary to
construct an FMO interaction diagram will be written to a separate
output file.

%%%%%%%%
\subsection{{\sf FCO} (optional)}

Perform Fragment Crystal Orbital analysis.
The lines following this keyword are identical to those for the {\sf
FMO} keyword.


%%%%%%%%
\subsection{{\sf Average Properties} (optional)}

Do an average properties calculation for the system.  This consists
of:

\begin{itemize}

\item Generating a total DOS curve.

\item Finding E$_f$, the Fermi energy.

\item Determining the average overlap population and reduced overlap
population matrices within the unit cell (if the printing option for
these is set).

\item Finding average orbital occupations and net charges.

\item Reporting the average values of any COOPs specified (see
keyword {\sf COOP}).

\item Show average occupations of Fragment MO's (if FMO analysis is
being done).

\end{itemize}

\noindent {\bf NOTE:} if you are doing an average properties calculation on an
extended system, you
{\em must} provide a set of K points, see keyword {\sf K Points}.

Specifying an average properties calculation for a molecular problem
will allow the generation of MOOP (Molecular Orbital Overlap
Population) diagrams and the RCM (Reduced Charge Matrix) as for
extended syatems.

%%%%%%%%
\subsection{{\sf No Total DOS} (optional)}

Turns off printing of the total DOS into the output file.  This option
can be used to conserve disk space when the total DOS isn't going to
be looked at.  

{\bf Note:}  If this keyword is specified neither total DOS nor
projected DOS calculations will be done.


%%%%%%%%
\subsection{{\sf Dump Overlap} (optional)}

Toggles creation of a binary file containing the overlap matrix at
each K point.  This file can be used with the {\tt matrix\_view} utility to
generate pictures of overlap matrices.

%%%%%%%%
\subsection{{\sf Dump Hamil} (optional)}

Toggles creation of a binary file containing the hamiltonian matrix at
each K point.  This file can be used with the {\tt matrix\_view} utility to
generate pictures of hamiltonian matrices.

%%%%%%%%
\subsection{{\sf Dump Dist} (optional)}

Toggles creation of a binary file containing the distance matrix
for the system.  This .DMAT file can be used by the utiltity
{\tt cooperate} to generate specifications for COOPs automatically.


%%%%%%%%
\subsection{{\sf Projected DOS} (optional)}

This section contains the list of densities of states (DOS's) that
will be projected.  Each DOS can have multiple contributions which
will be added up.

The keyword is followed by a line containing the number of different
projections: \pvar{num\_proj\_DOS}. The next \pvar{num\_proj\_DOS} lines
should consist of:

\pvar{type} \pvar{contrib1} \pvar{weight1}, \pvar{contrib2}
\pvar{weight2},...

\pvar{type} should be either {\tt Atom}, {\tt Orbital}, or {\tt FMO} to indicate
whether the contribution from an entire atom, a single orbital, or a
fragment MO is being projected out.

There can be as many contributions to each projection as
you like, just put it all on one line and separate the
\pvar{contrib}--\pvar{weight} pairs by commas.
Entries for a single projection can be spread over multiple lines by placing a ``$\backslash$'' at the end of each line.
To {\bf average} contributions make the sum of the individual
contributions add up to 1.0.  To {\bf add} contributions, make each of
the contributions 1.0.  In general, it is a good idea to add projected
DOS curves rather than average them.

%%%%%%%%
\subsection{{\sf COOP} (optional)}

Do a Crystal Orbital Overlap (or Hamilton) Population analysis \cite{eht2,cohp3,mulliken,coop}. {\bf Note:} the {\sf COOP} option results in a Molecular Orbital Overlap (or Hamilton) Population analysis when the {\sf molecular} keyword is specified.

\noindent The line following the keyword has the total number of COOPs
specified (not just the number of different types): \pvar{tot\_num\_COOPs}.

\noindent The next \pvar{tot\_num\_COOPs} lines should contain the definitions
of the COOPs themselves in the form:

\pvar{type} \pvar{which} \pvar{contrib1} \pvar{contrib2} \pvar{a}
\pvar{b} \pvar{c} \\

\noindent \pvar{type} should be either {\tt Atom}, {\tt Orbital} or {\tt FMO} to indicate
whether the COOP between atoms, orbitals or fragment MO's is being projected. \\

\noindent Setting \pvar{type} to {\tt H-Atom}, {\tt H-Orbital} or {\tt H-FMO} will generate the corresponding Hamilton population between atoms, orbitals or fragment MO's respectively. \\

\noindent \pvar{which} is the number of the COOP.  Multiple COOPs with the same
value of \pvar{which} will be averaged. \\

The COOP reported is between \pvar{contrib1} in the unit cell and
\pvar{contrib2} in a cell defined by the vector (\pvar{a} \pvar{b}
\pvar{c}).  For example, the following sample section will average 
the COOP between atoms 1 and 2 in the unit cell with that between atom
2 in the unit cell and atom 1 in the adjacent cell in the \pvar{b}
direction:

\shrinkspacing
\begin{verbatim}

COOP
; the total number of lines here:
2
; definitions of the COOPS
;type  which  contrib1  contrib2     a b c
Atom    1       1          2         0 0 0
Atom    1       2          1         0 1 0

\end{verbatim}
\resumespacing

\noindent {\bf Note:}  If you are doing a COOP calculation on a high-symmetry
system where you are using K points within the irreducible wedge of
the first Brillouin zone, it is very important that you average all
symmetry equivalent bonds \cite{thesis}.
If you do not do so, your results may be inaccurate.

%%%%%%%%
\subsection{{\sf Printing} (optional)}

This keyword controls what information is printed into the output
file.  In most cases, this also controls what things the program
actually calculates.  For example, if the user doesn't request that
the reduced overlap population matrix be printed, then there's no
reason to calculate it.  We have tried to make the program
``smart'' about what it calculates, but, there
may be problems here.  Please let us know if you see strange behavior.

This keyword is different from all the others in that the program
expects it to be followed by another list of keywords.  In fact, any
keyword following {\sf Printing} is assumed to be controlling what
gets printed.  The way to tell \calcprog\ that you are done giving it
printing options is to either let it hit the end of the file (i.e. to
have {\sf Printing} as the last keyword in your input file, or to put
the keyword {\sf End\_Print} at the end of the printing options. \\

\noindent Each of the printing keywords is described below.

\begin{itemize}

\item {\sf Distance:}
Print the distance matrix.

\item {\sf Overlap Population:}
Print the Mulliken overlap population matrix.

\item {\sf Reduced Overlap Population:}
Print the Mulliken reduced overlap population matrix.

\item {\sf Charge Matrix:}
Print the charge matrix.

%\item {\sf Reduced Charge Matrix:}
%Print the reduced charge matrix. {\bf NOTE:} this is not yet
%implemented.

\item {\sf Wave Functions:}
Print the wavefunctions for the molecule.

\item {\sf Net Charges:}
Print the net charges on the atoms, as determined using Mulliken
population analysis.

\item {\sf Overlap:}
Print the overlap matrix.

\item {\sf Hamil:}
Print the hamiltonian matrix.

\item {\sf Electrostatic:}
Print the electrostatic contribution to the total energy.  {\bf NOTE:}
this is under development and is not to be considered reliable.

\item {\sf Levels:}
Toggles the printing of energy levels at each k point in an extended
calculation. 

\item {\sf Fermi:}
Print the Fermi energy (this is primarily useful in combination with
the Walsh option, described below). 


\item {\sf Orbital Energy:}
Allows the energy of a particular orbital to be printed.  (this is
primarily useful in combination with 
the Walsh option, described below). 

\item {\sf Orbital Coeff:}
Allows the coefficient of a particular atomic orbital in a given
molecular orbital to be printed.  (this is
primarily useful in combination with 
the Walsh option, described below). 

\item {\sf Orbital Mapping:}
Generates the scheme used to number the individual atomic orbitals in a calculation (especially useful when working out the contributions for COOP's)

\item {\sf Levels}
Print out the calculated energy levels at each k point in an extended
calculation. 
\end{itemize}

Each of these options control printing at every K point and/or step
along a reaction coordinate.  Turning on all the printing options can
lead to a {\bf huge} output file if you have a lot of K points or
steps in a Walsh diagram.

Placing the keyword {\sf Transpose} after a printing option for a
matrix will result in the transpose of the matrix being printed.  This
feature has been introduced to appease those who think that the default
way of printing is stupid.

To facilitate the construction of graphs of various quantities
(overlap populations, net charges, etc.) along a reaction coordinate,
there is a second option that can be used with printing options.  If
you place the keyword {\sf Walsh} on the same line as a printing
option, then you can select a particular quantity to monitor along the
Walsh diagram.  These values are put into a separate file.  If you are
using an input file named {\tt foo.bind} then the values of these
quantities would be put into {\tt foo.bind.walsh}.

To use this feature, place {\sf Walsh} after your
printing keyword, then on the next line put the following three pieces
of information:

\pvar{type} \pvar{contrib1} \pvar{contrib2}

Once again, \pvar{type} is one of {\tt Atom}, {\tt Orbital} or {\tt FMO} and
\pvar{contrib1} and \pvar{contrib2} refer to particular atoms,
orbitals or fragment MO's.

{\bf NOTE:} there are certain combinations of these Walsh printing
options which do not make sense.  For example, specifying that you
want to monitor the reduced overlap population between 2 orbitals is
nonsensical.  The program will notice this and complain, so just think
a bit before you start printing everything out.

For example, the following section would print out the entire overlap
population matrix and all the net charges at every step into the main
output file, and then
print the overlap population between orbitals 13 and 23 into the Walsh
output file:

\shrinkspacing
\begin{verbatim}
; start dealing with printing options
Print

Overlap Population
Net charges

; this is a Walsh printing value
Overlap Population   Walsh
Orbital 13 23

End_Print
\end{verbatim}
\resumespacing


%%%%%%%%
\subsection{{\sf MO Print} (optional)}

This keyword controls creation of a .MO file.  This can be read in by
\viewprog\ to produce iso-surface plots of molecular and crystal
orbitals.

The first line following the keyword contains the number of MO's to be
printed: \pvar{num\_MOs}.  The next \pvar{num\_MOs} lines contain the
numbers of the individual MO's that should be printed.

If you are doing an extended calculation, the MO's will be printed at
each k point.  If you are doing a Walsh diagram, the MO's will be
printed at each step along the reaction coordinate.  


%%%%%%%%
\subsection{{\sf Orbital Occupations} (optional)}
This section is used to change the occupations of molecular orbitals.
This is primarily useful for trying to model open shell systems or
molecules in excited states.  

The first line following the keyword should contain the number of
orbital occupations to change, \pvar{num\_occups}.  The next
\pvar{num\_occups} lines consist of an integer specifying the orbital
whose occupation should be changed and a real number specifying what
the new occupation should be.

For example, the following piece of an input file would place 1
electron in both orbitals 49 and 50:

\shrinkspacing
\begin{verbatim}

Orbital Occupations
; the number of occupations to change
2
; the orbitals and new occupations
49 1.0
50 1.0
\end{verbatim}
\resumespacing

%%%%%%%%
\subsection{{\sf Charge Iteration} (optional)}

\calcprog\ has the capability to perform charge iteration
(a self consistent adjustment of the H$_{ii}$s in order to lessen the
amount of charge flow).  
The charge iteration algorithm in \calcprog is somewhat experimental (... so beware!) and is discussed in the file {\bf charge.ps}.

For those of you who know no fear here's a summary of the CI keywords:

The keywords controlling the CI process are sandwiched between the {\sf charge iteration} and {\sf end charge} keywords.

The keywords within the CI block are: 

{\sf Param}
followed by a line giving the number of different atoms that parameters
will be specified for: \pvar{num\_CI\_parms}
each of the next \pvar{num\_CI\_parms} lines should contain the charge
iteration parameters in the form:
\pvar{Atomic\_symbol  $s_A s_B s_C p_A p_B p_C d_A d_B d_C$}

(i.e. the A, B, and C parameters for each of the orbitals.  These are
the same parameters used in the old programs for single configuration
CI and are NOT distributed with YAeHMOP).
{\bf Note:} charge iteration for f orbitals is not supported.

{\sf Vary}(follows Param)
followed by a single line containing the numbers of the atoms
whose parameters are to be varied. This is a comma delimited list, you
can use ``-'' to abbreviate series of atoms (i.e. 1-4 and 1,2,3,4 are
equivalent)

{\sf Tolerance}
followed by a single line specifying the tolerance used to terminate the iteration

{\sf lambda}
followed by a single line specifying the step size for the iteration process

{\sf max iter}
followed by a single line specifying the maximum number of iterations


%%%%%%%%
\subsection{{\sf Sparsify} (optional)}

This is used to set small elements of the hamiltonian and overlap
matrices to zero.  
The next line should contain the value which is considered to be zero.

{\bf NOTE:} This is primarily here for development purposes and 
we'd encourage you not to use this.

%%%%%%%%
\subsection{{\sf Just Average E} (optional)}

Tells \calcprog\ to only generate the average energy, total DOS, and
Fermi level of
the system.  This causes the program to require considerably less
memory when run on systems with a lot of orbitals.

If you are using a version of \calcprog\ that uses the
LAPACK libraries to diagonalize the matrices, then only 
eigenvalues will be generated.  This can result in a significant speed
increase.  (A factor of 10 decrease in execution time for
sufficiently large systems is possible !).

%%%%%%%%
\subsection{{\sf Just Matrices} (optional)}

Generate just the overlap and hamiltonian matrices, then exit.  This
is basically useless unless it is used in conjunction with either the
{\sf Dump Overlap} or {\sf Dump Hamil} keywords, or the {\sf
Hamiltonian} or {\sf Overlap} printing options.

%%%%%%%%
\subsection{{\sf Diagwo} (optional)}

Performs the matrix diagonalization without using the overlap matrix,
so that a calculation is similar to a simple H\"uckel calculation.


