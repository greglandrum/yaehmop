\chapter{Using \viewprog}

\viewprog\ is written to display results on either X Windows displays
or Tektronix terminals.  The program will automatically detect whether
or not X Windows are available and will use them if they are.
Printing is handled by generating Postscript 
files which can then be directly printed or included in documents.
The actual graphics calls used to draw the data are all included in a
separate file, so it should be reasonably easy to port the program to
other graphics systems.

Since the use of the X and Tek versions of \viewprog\ is different,
they will be dealt with separately.

{\bf Note:}  I haven't put much work into the command line (Tektronix)
version of \viewprog\ recently, so it doesn't have a lot of the
features mentioned below and some of the features it purportedly does
have may not work.

\section{Using \viewprog\ in X}

When you start \viewprog\ under X, it will open 2 windows.  The first,
and larger, window (the graphics window) is used to display output.
The second window (the main button window) has buttons which are used
to control the program. 

The individual button windows and the functions of the buttons found
therein are described below.  Each button is only described once, so
though many different windows have a {\bf X Legend} button, it is only
described once. 

You can cause the program to redraw at any time (except when an
isosurface is being evaluated) by middle clicking in any of the
windows.

\subsection{Special keys in \viewprog}

There are a number of keys that can be used in \viewprog, some of
these duplicate features found in button windows, some are unique.

\begin{itemize}
\item {\bf q}: will cause \viewprog\ to quit.
\item {\bf r}: switches into rotate mode.
\item {\bf t}: switches into translate mode.
\item {\bf c}: switches into center mode.
\item {\bf s}: switches into scale mode.
\item {\bf the spacebar}: switches into choose mode.
\item {\bf h}: in choose mode hides the selected atoms
\item {\bf +}: in choose mode shows previously hidden atoms
\item {\bf z}: rotates the selected molecule so that you are looking
along the Z axis.
\item {\bf y}: rotates the selected molecule so that you are looking
along the Y axis.
\item {\bf x}: rotates the selected molecule so that you are looking
along the X axis.
\item {\bf 1}: \viewprog\ will prompt you for a file name to use, then
write an input file for {\tt rayshade}.
\item {\bf d}: writes the cartesian coordinates of the molecule to
standard output.
\end{itemize}


\subsection{General use of buttons}

In order to avoid having to use a user interface library that may not
exist on some machines, I wrote all the button code myself.  This
means that the buttons aren't necessarily the most beautiful things
you've ever seen, and sometimes they behave in ways which are just
plain wrong (for example, text can overflow out of the button).  The
most important thing from my perspective is that these buttons do
work, and the code to deal with them is simple and small.

To activate a button, left click on it.  If it is a toggle button,
then the toggle will be changed.  If the button is for changing the
value of some variable, you will be prompted to enter a new value for
that variable (no, I didn't write dialog box code).  

Buttons which are for toggling the display of lines will show a sample
of the line style to the right of the button.  You can change this
line style by right clicking in the toggle button.

\subsection{The main button window}

The buttons in the main button window are described below:

\begin{itemize}

%%%%%%%%%%%%%%%%%%
\item {\bf the Mode button}:  This is the top button in the window.
It displays what mode the program is using to manipulate the graphics
displayed in the graphics window.  Left clicking in this window
changes the active mode. This mode determines what action
the control keys have.  The control keys are i,j,k,l,p,and ;.  The
first four of these form an inverted arrow on a standard keyboard.  

The possible modes are:

\begin{itemize}
\item None:  the control keys do nothing.

\item Rotate:  the control keys rotate the active object.
i and k rotate about the y axis, j and l rotate about x, and p and ;
rotate about z.  Holding down shift while hitting any of these keys
results in a larger rotation step.  
In versions of \viewprog\ greater than or equal to 2.0, you can also
rotate molecules in rotate mode by left clicking and dragging in the
graphics window.  This is easier to do than it is to explain, so try
it out.
{\bf Note:}  Rotations only apply to displayed molecules and MO's, not
to graphs because that would be silly.


\item Translate:  the control keys translate the active object.
i and k move along the y axis, j and l move along x, and p and ;
move along z.  Holding down shift while hitting any of these keys
results in a larger step.  

\item Center:  the control keys translate the center of the active
object. 
i and k move along the y axis, j and l move along x, and p and ;
move along z.  Holding down shift while hitting any of these keys
results in a larger step. 
This is different from the {\sf Translate} mode for molecules and MO
surfaces in that the molecule and the point that the camera used to
construct the perspective view looks at are move simultaneously.  This
allows the molecule to be moved about the screen without the view
changing. 
In versions of \viewprog\ greater than or equal to 2.0, you can also
change the center of your molecule by left clicking in the graphics
window.  When you do this, the center of the molecule will be moved to
where you clicked.  If you then drag, the molecule will move.
{\bf Note:}  For anything other than molecules and MO surfaces {\sf
Translate} and {\sf Center} are equivalent.

\item Scale: the control keys change the size of the active object. 
j shrinks along x, l grows along x, k shrinks along y, i grows along
y, ; shrinks along z, and p grows along z.  Once again, holding down
shift while hitting a key increases the size of the step.

\item Choose: Left clicking on atoms selects them.  Right clicking
then either displays the distance between those atoms (two selected),
the angle between them (three selected), or the relevant dihedral
angle (four selected).  A label is placed at the location
of the right click and lines are drawn to the controlling atoms.  If
you right click with just a single atom selected, the coordinates and
identity of that atom will be printed in the window from which you ran
\viewprog.  This is a convenient way to find the coordinates of a
particular atom if you forget (or if you have centered the molecule
using the ``center'' button in the molecule option window).
The labels displayed in Choose mode remain on screen until the {\sf
Clear Labels} button is 
hit.  

\end{itemize}

%%%%%%%%%%%%%%%%%%
\item {\bf Read Molecule}:  Reads in the atomic positions of a
molecule.  You will be prompted for the name of the {\bf output} file
containing the geometry.  Enter the name of the file in the window
from which you started up \viewprog.  The molecule will be read in and
displayed and a molecule button window will be opened.

%%%%%%%%%%%%%%%%%%
\item {\bf Read MO}: Reads in the specification of an MO.  You will be
prompted for the name of the {\bf input} file used to perform the
calculation. Enter the name of the file in the window from which you
started up \viewprog.  The program will read in the geometry of the
molecule itself from the {\tt .out} file and the MO specification from
the {\tt .MO} file.  You will be prompted for which MO you which to
use if there are multiple MO's in the {\tt .MO} file.  {\bf Note:}  To
use this option you must have specified MO printing when \calcprog\
was run.   

%%%%%%%%%%%%%%%%%%
\item {\bf Read Contours}:  Reads in a contour plot.  You will be
prompted for the name of the file containing the contour data.
This option is primarily intended for dealing with FCO plots.

%%%%%%%%%%%%%%%%%%
\item {\bf Read FMO}:  Reads in FMO data and constructs an interaction
diagram.  You will be prompted for the name of a {\tt .FMO} file.
\viewprog\ will read in the FMO data, construct an interaction
diagram, and open an FMO options window.

%%%%%%%%%%%%%%%%%%
\item {\bf Read Props}:  Reads in average properties (DOS, COOP or
COD) data and displays it.  You will be prompted for the name of a
data file (either {\tt .DOS}, {\tt .COOP}, or {\tt .SUB}).  The
program will read in the data and open a property options window.

%%%%%%%%%%%%%%%%%%
\item {\bf Read Walsh}:  Reads in data for a Walsh diagram.  You will
be prompted for the name of a {\tt .WALSH} file.  \viewprog\ will read
in the data and open a walsh option window.

%%%%%%%%%%%%%%%%%%
\item {\bf Read Bands}:  Reads in the data to display a band
structure.  You will be prompted for the name of a {\tt .bands} file.
\viewprog\ will display and band structure and open a bands option
window.


%%%%%%%%%%%%%%%%%%
\item {\bf Read Graph}:  Reads in raw data for a graph.  You will be
prompted for the name of the file containing the graph data.  This
allows construction of very primitive graphs.  This option just exists
because it was easy to do.  \viewprog\ is not intended to be a general
purpose graphing program.

%%%%%%%%%%%%%%%%%
\item {\bf Fill Proj.s}:  Toggles filling of projected DOS curves.  On
screen these will be shown shaded, in the printed output they will be
lined.  {\bf Note:} in the Macintosh version of \viewprog, this
keyword has no effect on the way things are displayed on screen, it
does still affect the Postscript output.

%%%%%%%%%%%%%%%%%%
\item {\bf Purge!}: Deletes all currently displayed objects and closes
all of their button windows.

%%%%%%%%%%%%%%%%%%
\item {\bf Printing Options}:  Opens a window which allows you to
change some of the default behavior for the Postscript printing.


%%%%%%%%%%%%%%%%%%
\item {\bf Print}:  Prompts for the name of a file, then redraws the
screen, writing its contents to the file as Postscript.  This file can
then be printed however you normally print {\tt .ps} files.

%%%%%%%%%%%%%%%%%%
\item {\bf Clear Labels}:  Removes labels from the display.  They
will not be refreshed, so once you clear them, they are gone.

\end{itemize}


%%%%%%%%%%%%%%%%%%
%%%%%%%%%%%%%%%%%%
\subsection{The PS options button window}

This window allows you to control some of the default behavior of the
Postscript printing from \viewprog.  

\begin{itemize}
\item {\bf Location}:  This button controls where the graph is located
on the output page.  There are three possible values: Top, Middle and
Bottom.  The default is Bottom.

\item {\bf Font}: Allows specification of the standard text font.  The
default is Times--Roman.  {\bf Note:}  This option can be overridden
using the enhanced Postscript commands (described in an Appendix).

\item {\bf Font Size}: Allows specification of the standard text font
size (in points).  The default is 12.

\item {\bf Scale}: Allows scaling of the whole output.  This does {\em
not} affect the size of the text font, to change that use the {\bf Font
Size} button.

\end{itemize}

{\bf Note:} None of the changes
made in this window will be visible on screen.


%%%%%%%%%%%%%%%%%%
%%%%%%%%%%%%%%%%%%
\subsection{The molecule button window}

This is the window which is popped up when a molecule is opened.  It
is used to control the viewing options for the molecule being
displayed.  

\begin{itemize}
\item {\bf Hydrogens?}: Toggles drawing of hydrogens.

\item {\bf Dummies?}: Toggles drawing of dummy atoms.

\item {\bf Center}:  When this is pressed, the molecule is moved so
that it is centered at the center of mass ({\bf Note:} since
\viewprog\ doesn't actually know what the masses of your atoms are,
the center of mass is actually calculated assuming all the atoms have
the same mass.  The resulting positioning is usually still right.).


\item {\bf Hide Atoms} Allows you to make it so that some atoms in the
molecule aren't displayed.  You will be prompted for a list of atoms
to hide.  Enter a comma delimited list. You can use the dash (hyphen)
just as it was used in the FMO specification.  For example, the
following list will hide atoms 1-23 and 99:
{\tt 1-23, 99}

\item {\bf Show Atoms} Allows you to ``unhide'' atoms you may have hidden
before. 

\item {\bf Axes?}: Toggles display of a set of axes on screen.

\item {\bf Outlines?}: Toggles the drawing of dark circles around
the atoms being displayed.

\item {\bf Shading?}: Toggles shading of the atoms being displayed.

\item {\bf Crosses?}: Toggles display of pseudo-3D crosses on the
atoms.  If this is turned on and shading is turned off, the interiors
of the atoms are drawn as solid white with the cross superimposed.

\item {\bf Connectors?}: Toggles drawing of lines connecting atoms
which are a distance within \pvar{bond\_tol} (see below) of the sum of
their covalent radii apart.

\item {\bf Fancy Lines?}: When this is on lines between atoms are
drawn as if they are intersecting the sphere of the atoms.  When off,
the lines are drawn all the way to the center of the atoms.  Turning
``Fancy Lines'' off is useful when drawing a structure without the
atoms being displayed.

\item {\bf Breaking Lines?}: When this is turned on, lines ``cut'' those
behind them that they intersect.

\item {\bf Tube Lines?}:  Toggles display of bonds as tubes instead of
lines.  Tubes are drawn as a white center with a black edges and 
cut lines behind them just like breaking lines.  {\bf Note}: if both
``Tube Bonds'' and ``Breaking Lines'' are turned on, only the Breaking
Lines will be drawn.


\item {\bf Numbers?}: Toggles display of the numbers of atoms.

\item {\bf Symbols?}: Toggles display of atomic symbols.

\item {\bf Line Width}: Used to control the thickness of the lines
drawn between atoms.

\item {\bf Bond Tol}: Used to enter a new value of \pvar{bond\_tol}.
When this is changed the lines between atoms are recalculated.

\item {\bf Rad Scale}: Used to control the scaling of the circles
drawn for atoms.

\item {\bf Grow Xtal!}:  This button only appears for molecules that
have lattice parameters in the output file (i.e. extended systems).
Clicking this allows you to show more than one unit cell.

\end{itemize}

%%%%%%%%%%%%%%%%%%
%%%%%%%%%%%%%%%%%%
\subsection{The MO button window}

{\bf Note:} This section is substantially changed since \viewprog\
version 1.2.

When an MO isosurface is opened, two button windows are opened.  One
of these windows is a Molecule window as described above.  The other
window is used to control the drawing of MO isosurfaces on
screen.  These options are described below.

There are two ways to draw isosurfaces in \viewprog\ version 2.  The
first method draws the isosurface just as it was done in version 1.2:
a solid isosurface made up of filled triangles is drawn.  These
isosurfaces do not look particularly good when drawn in viewkel, but
generate very nice illustrations when {\sf rayshade} is used to
raytrace the object. The second method draws Jorgenson and Salem style
contour plots of the surface, with hidden line removal. These plots
are similar to those generated by {\sf CACAO} and {\sf PSI88}.  I will
refer to these as ``solid'' surfaces and ``contour'' surfaces
respectively.  Finally, this button window is also used to generate a
two dimensional contour plot of an MO, I'll refer to these as ``contour plots''.

While some buttons apply to both types of figures, the majority
control only solid {\em or} contour surfaces.

To explain what the solid surface buttons do, it is necessary to
understand how the calculation of solid isosurfaces is done.  The algorithm
used to generate the polygonal isosurface is taken from a section by
Jules Bloomenthal (at Xerox PARC) in the book {\em Graphics Gems IV}.
The algorithm as published is suitable for polygonalization of
continuous surfaces.  Unfortunately, isosurfaces of molecular and
crystal orbitals are rarely continuous.  They are, instead, made up of
a number of discrete parts (usually lobes centered on particular
atoms).  In order to work around this problem, Bloomethal's algorithm
has been modified to look for individual pieces of the surface around
each atom in the molecule.  The algorithm starts at the corners of a
cube of side length \pvar{search\_radius} centered on each atom.  The
default value of \pvar{search\_radius} seems to work most of the time,
if you see that lobes are obviously missing, try changing the value of
\pvar{search\_radius}.   The algorithm evaluates the MO values on a
grid of spacing \pvar{voxel\_size}.  Increasing \pvar{voxel\_size}
results in quicker evaluation and drawing of the surface (less
triangles are found, so the surface can be drawn more rapidly), but
the resolution suffers.  Decreasing \pvar{voxel\_size} gives a
smoother surface, but it takes much longer to calculate and to draw.
My best advice here is to start with the default value (0.2\AA) and
then play around with it to see what size suits your purpose.


{\bf Some Cautions about solid isosurfaces:} 

\begin{itemize}

\item I haven't had a
chance to do shading (simulated lights) of the isosurfaces.  I
think that this will improve the appearance of the final images.  

\item A heuristic needs to be developed to warn the user when lobes of
the MO might have been missed.  I have some ideas here, but I'm still
working on them.

\end{itemize}

When the surface is first read in, the user is prompted for values
used in constructing the radial lookup table.   This is a table of
values for the radial part of the wavefunctions.  Use of this lookup
table results in a tremendous gain in speed.  The default values
should be fine for almost any calculations.

One important note about {\em both} solid and contour isosurfaces: 
\viewprog\ does not deal with the imaginary part of 
wavefunctions yet.  I will put this in later, until then, limit
yourself to pure real wavefunctions.  You'll always be okay with
molecular wavefunctions, but general k points for extended systems
will be problematic.

Well, with my salvo of disclaimers fired, here are the explanations of
the buttons in the MO options window.

\begin{itemize}

\item {\bf Surface?}: Toggles display of the solid isosurface once it has
been calculated.

\item {\bf Molecule?}: Toggles display of the molecule.

\item {\bf Isosurface}: This is the isosurface value that is displayed
on screen.  This is applies to both solid and contour isosurfaces.

\item {\bf Voxel Size}: The spacing of the grid used to calculate the
solid surface.

\item {\bf Search Rad}: The size of the cube around each atom used to
provide starting points for the solid surface search.

\item {\bf Slop}:  Used to determine how far past the end of the
molecule's bounding box the solid isosurface grid extends.

\item {\bf Exclude Atoms}:  Used to remove atoms from the calculation
of the isosurface.  For example, if you are doing a 7 layer metal slab
and are only interested in the surface states (a topic near and dear
to my heart), you can exclude the atoms which are in the middle of the
slab.  This will result in a cleaner looking, more quickly evaluated
isosurface.  You will be prompted to enter a list of atoms, the syntax
here is the same as for the Hide Atoms button described in the
Molecule options section.

\item {\bf Include Atoms}:  This is used if you make a mistake when
you exclude atoms. It turns the atoms you switched off back on.

\item {\bf Surf Evolve}: Starts the calculation of a solid isosurface.

\item {\bf Change MO}:  Allows you to change which MO you are looking
at without having to reload the whole molecule.

\item {\bf Tri\_Shading?}: Toggles filling of the triangles which make
up the displayed solid isosurface.

\item {\bf Tri\_Outlines?}: Toggles the drawing of outlines around the
triangles which make up the solid isosurface.

\item {\bf Contours?}: Toggles display of the calculated contours that
make up either a contour plot of an MO or a contour isosurface.

\item {\bf Contour Mode}:  This mode button determines the method
which is used to generate contour levels in contour plots.  There are
three possibilities: {\sf Auto} \pvar{num\_contours} contour levels
are automatically selected between the maximum and minimum values in
the data set; {\sf Incremental}, you will be prompted for a starting
value and a step between contours when the contour plot is evaluated;
{\sf Discrete}, you will be prompted to enter the \pvar{num\_contours}
contour values when you evaluate the contour plot.  

\item {\bf Contour it}: Evaluates a contour plot.  You will be
prompted for:
\begin{itemize}
\item  the orientation of the plane used to evaluate the contours
(i.e. perpendicular to X, Y or Z)
\item  the height and width of the plane in which the MO will be
evaluated.
\item the number of steps to take along each side of the evaluation
plane.
\item the number of contours you want to use
\item an offset, this shifts the evaluation plane off the origin,
useful if you want to contour an MO where most of the density isn't at
X,Y, or Z = 0.
\item Whether or not you want to do a ``stack'' of contour plots.
  If you say yes here, you will be prompted for the number of planes
you want in the stack, the location of the start of the stack, and the
distance between planes in the stack.  What this does is generate a
series of plots which are parallel to each other and located at
different heights.  This is a quick and dirty way to get a feeling for
the shape of an MO.  {\bf Note:}  if the contour mode is {\sf Auto},
\viewprog\ will generate the contour levels automatically in the first
plane of the stack, but then will use those contour values in all
other planes.
\end{itemize}

\viewprog\ will then go off and evaluate the values of the MO in the
plane, contour that data, show you the values of the contours it
found, and finally display those contours (if the {\sf Contours?}
toggle described above is switched on).  MO contour plots are real
three dimensional objects just like the molecule they are associated
with, and will rotate, translate, and scale in three dimensions along
with the molecule.  This feature seems gratuitous, but it's kind of
fun to play with and it was a lot easier to implement than having the
contour be a fixed 2D object.  It can even be useful:  if you rotate
the molecule so that you are looking at the edge of the contour plot,
you can see exactly where it cuts through your molecule.

\item {\bf Contour Surf}: Evaluates an MO contour plot.

\item {\bf Invert Phase}:  This toggle allows you to invert the phase
of the MO displayed on screen.  This can be helpful when you want to
visually compare two similar MO's that have opposite phases: just
invert the phase of one of them.

\item {\bf Hidden}:  Toggles the use of the hidden line removal
algorithm for contour plots and MO surfaces.  When \pvar{Hidden} is
on, you should no longer be able to ``see through'' the MO isosurface.
However, this can make it take a lot longer to draw plots, so you may
want to have \pvar{Hidden} turned off when you are rotating your
molecule.  The hidden line removal does not work particularly well for
normal contour plots, but \viewprog\ will still allow you to use it if
you wish. {\bf Note}: there are still some boundary conditions that
get screwed up sometimes in the hidden line removal.  If a plot looks
bad: rotate the molecule a small amount, that should clear things up.
{\bf Very, Very Important Note}:  The hidden line removal algorithm is
not guaranteed to work at all if you have some contours displayed
which are not closed (this can arise if the plane in which you
evaluated the MO was too small).   Sometimes it works, sometimes it
does stuff that is very, very wrong.  

\item {\bf Grow Xtal}:  This grows the crystal and propagates the MO
coefficients with the correct phase factors for the given k point.  If 
you want to display the MO of an extended system in more than one unit
cell, you {\bf must} use this button to grow the crystal.  Using the
{\bf Grow Xtal} button in the Molecule button window will not
correctly propagate the MO coefficients.

\end{itemize}

{\bf Note:}  If are looking at an extended system, and you want to see
the crystal orbital in more than the home unit cell, you must
evaluate the isosurface after growing the crystal.


%%%%%%%%%%%%%%%%%%
%%%%%%%%%%%%%%%%%%
\subsection{FMO button window}

This window contains buttons for controlling the display of an FMO
interaction diagram.

\begin{itemize}
\item {\bf Electrons}:  Controls the drawing mode for the electrons in
the interaction diagram.  Each mode is described below:

\begin{itemize}
\item None: Do not draw electrons.

\item HOMO: Draw electrons only in the highest occupied level of each
fragment and the molecule.

\item All: Draw electrons in all occupied levels on screen.
\end{itemize}

\item {\bf Levels}: Toggles display of the levels in the FMO diagram.

\item {\bf Left Fragment}: Controls which fragment is displayed on the
left side of the interaction diagram.  Set this to zero or -1 to not
show a fragment on the left side.

\item {\bf Right Fragment}: Controls which fragment is displayed on the
right side of the interaction diagram.  Set this to zero or -1 to not
show a fragment on the right side.

\item {\bf Y min}: This is the minimum energy displayed.

\item {\bf Y max}: This is the maximum energy displayed.

\item {\bf Level Width}: The width (horizontal length) of the levels
drawn.

\item {\bf Thickness}: The thickness of the lines used to draw the
levels.

\item {\bf Electron len}: The length of the lines used to represent
electrons.

\item {\bf Y tics}: Toggles the display of tic marks on the y axis.

\item {\bf Frame}: Toggles the display of a frame around the
interaction diagram.

\item {\bf Show Title}: Toggles display of the title of the graph (if
one has been entered).

\item {\bf Title}: Allows you to enter a new title to be displayed
across the top of the graph.

\item {\bf Main Label}: Allows you to enter a new label for the
central set of levels (those of the full molecule).

\item {\bf Frag {\sf n} Label}: There will be one of these buttons for
each fragment used in the calculation.  These are used to enter labels
to be displayed under each fragment displayed.

\end{itemize}

%%%%%%%%%%%%%%%%%%
%%%%%%%%%%%%%%%%%%
\subsection{Walsh button window}

This window is used to control the display of Walsh diagrams.  The
coordinate used to label the x axis is specified when {\sf fit\_walsh}
is run.

\begin{itemize}

\item {\bf MO's}: Toggles display of the MO levels.

\item {\bf Total E}: Toggles display of the total energy.

\item {\bf X tics}: Toggles display of X tics.

\item {\bf MO tics}: Toggles display of Y axis tic marks (for the
energies of the MO's displayed) on the left side of the diagram.

\item {\bf Tot E tics}: Toggles display of Y axis tic marks (for the
total energy curve) on the right side of the diagram.

\item {\bf X Legend}: Allows you to enter a legend to be displayed
under the X axis.

\item {\bf Y Legend}: Allows you to enter a legend to be displayed
beside the Y axis.
\end{itemize}

%%%%%%%%%%%%%%%%%%
%%%%%%%%%%%%%%%%%%
\subsection{Band button window}

Controls the display of band structure data on screen.

\begin{itemize}
\item {\bf Bands}: toggles display of the bands.

\item {\bf Show Fermi}: toggles display of the Fermi energy on the
band graph.

\item {\bf Fermi E}: allows you to enter a value for the Fermi energy.
The location of the Fermi energy is not stored in the band file, so
you must enter this yourself to get an accurate value.
\end{itemize}

%%%%%%%%%%%%%%%%%%
%%%%%%%%%%%%%%%%%%
\subsection{Properties button window}

This is used to control display of average properties data (DOS, COOP,
or COD data).  

\begin{itemize}
\item {\bf Curve n}: where n is an integer.  This is a toggle to
control display of each curve in the file.  The numbering of the
curves varies from type to type:  
\begin{itemize}
\item DOS data:  Curve 1 is the total DOS.  Projections are
labeled starting from Curve 2.  The projections are in the order in
which they were specified in the input file.

\item COOP data: The curves are numbered in the same way as the COOPs
were numbered in the input file.

\item COD data: There is only one curve per COD file, this is curve 1.
\end{itemize}

\item {\bf Integration n}: where n is an integer.  These toggle
display of
the integration of the corresponding curve.

\item {\bf Integ Scale}: toggles use of the integration data to label
the X axis.  This is particularly useful for integrated DOS data.
There are problems with using this option for COOP and COD data.
These will be fixed in the next version of the program.

\item {\bf Fermi E}: allows you to change the value of the Fermi energy.
For DOS and COOP data the value of the Fermi energy is stored in
the output file, so \viewprog\ will know the proper value.  

\end{itemize}

%%%%%%%%%%%%%%%%%%
%%%%%%%%%%%%%%%%%%
\subsection{Graph button window}

This controls displays of graph data.  There are no new buttons for
control of graph data.

%%%%%%%%%%%%%%%%%%
%%%%%%%%%%%%%%%%%%
\section{Using \viewprog\ from a Tek terminal}

When you are sitting at a Tek terminal (Tek 4014 compatible),
\viewprog\ will use a command line driven interface and will use the
graphics capabilities of the terminal to display graphs. 

You can get help in the Tek version of the program by typing ``help''
or ``?'' at the prompt.  This will list all the commands along with a
brief description.  Since this is available, I will not describe all
the features here.

Though it can be convenient, the Tek interface does have some
limitations: 

\begin{itemize}
\item You can not look at either molecules or MO's.

\item Not all features of each type of graph are modifiable.

\item Once a new graph has been opened, previously opened (and
displayed) graphs cannot be modified.

\item Due to limitations of the technology, the graph you see onscreen
is not nearly as similar to what comes out of the printer as in
the X windows version of \viewprog.

\end{itemize}

{\bf Note:}  It has been a long time since I have worked on the
Tektronix version of \viewprog, so it's not guaranteed to work at all.

%%%%%%%%%%%%%%%%%%
%%%%%%%%%%%%%%%%%%
