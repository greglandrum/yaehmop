\documentstyle[12pt,fullpage,misc,titlepage]{report}

\newcommand {\progname}[1] {\mbox {\bf {\sf #1}}}
\newcommand {\Hii} {\mbox {H$_{ii}$}}
\newcommand {\Hiip} {\mbox {H$_{ii}^{(k+1)}$}}
\newcommand {\Hiik} {\mbox {H$_{ii}^{(k)}$}}

\newcommand {\pvar}[1] {\mbox {\sl #1}}

\newenvironment{sampfile}
{ \baselineskip=12pt 
\begin{verbatim}}
{\end{verbatim}}

\begin{document}

\begin{center}
{\huge Charge Iteration in \calcprog }

{\large Greg Landrum}

{\large \today}

\end{center}


\baselineskip = 20pt

%%%%%%%%%%%%%%%%%%%%%%%%%%%%%%%%%%%%%%%%%%%%%%%%%%%%%%%%%%%%%%%%%%%%
\section{Goals and Problems}

The basic idea of charge iteration is to use calculated charges to
adjust the diagonal elements of the hamiltonian (the \Hii's) in a
self-consistent manner.  The main reason I see for doing this is to
have parameters that are reasonable for a given system that are
generated in a well defined manner (instead of having a different {\em
ad hoc} set of parameters for every calculation).  The way this is
done is using the equation:
\begin{equation}
\Hiip = A (Q^{(k)})^{2} + B Q^{(k)} + C
\label{eqn:basic_chg_it}
\end{equation}
where \Hiip\ is the new value of \Hii\ for step $k+1$, $A,B$ and $C$ are 
parameters and $Q^{(k)}$ is the net charge on the atom calculated
using the value of $\Hii^{(k)}$.

The traditional problem with charge iteration, as I see it, is that
the self consistent calculations do not always converge in a timely
fashion.  Some don't converge at all.  Since the form of the
expression for \Hiip (Equation \ref{eqn:basic_chg_it}) seems simple
and unlikely to lead to numeric explosions, I would say that the
convergence problems may be due to the implementation of the charge
iteration procedure.

For the sake of clarity (and because I haven't implemented the other
scheme), I'm going to limit myself to only discussing charge iteration
using a single configuration (a single set of parameters).

\section{The Old Way}

The implementation of charge iteration in \progname{ICON}, which is
what we've used, is pretty complicated.  Here's how it works.

The underlying form is that of equation \ref{eqn:basic_chg_it}, but
the calculation of 
the value of $Q$ is complicated.  First of all, the occupancy of each
orbital ($p_i$ where $i$ indexes the orbital) is calculated using a
damping scheme: 
\begin{equation}
p_i^{(k+1)}(initial) = p_i^{(k)}(initial) + \lambda \Delta p_i^{(k)}
\label{eqn:orb_occup_def}
\end{equation}
again $i$ indexes the orbital, $k$ indicates which step we're on,
$(initial)$ refers to the occupancy at the beginning of the step (not
the occupancy calculated using the results from this step), $\lambda$
is a step size, and $\Delta p_i^{(k)}$ is defined as:
\begin{equation}
\Delta p_i^{(k)} = p_i^{(k)}(final) - p_i^{(k)}(initial)
\label{eqn:delta_def}
\end{equation}
In equation \ref{eqn:delta_def} $(final)$ refers to the orbital
occupancy which was calculated in this step.

These orbital occupancies are used to calculate the net charge on each
atom $\nu$:
\begin{equation}
Q_{\nu}^{(k+1)} = \sum_{i\ on\ \nu} p_i^{(k+1)}(initial)
\label{eqn:old_net_chg_def}
\end{equation}
and these $Q_{\nu}^{(k+1)}$'s are used to generate the \Hiip's:
\begin{equation}
\Hiip = A_i (Q_{\nu}^{(k+1)})^2 + B_i Q_{\nu}^{(k+1)} + C_i  
\label{eqn:Hiip_def}
\end{equation}

The new \Hiip's are used to generate a new hamiltonian and the process
repeats until some convergence criterion is met.

The presence of the $\lambda$ (step size) in equation
\ref{eqn:orb_occup_def} is to provide some damping for the adjustment
(note that $\lambda < 1$).  This lessens problems with \Hii\ values
``sloshing'' between 2 values. 
In \progname{ICON}, the value of $\lambda$ is adjusted as the
calculation proceeds.  This is supposed to speed convergence.  The
formula for adjusting $\lambda$ is fairly complicated, so I'm not
going to go into it here.

\section{The New Way}

The way charge iteration is currently done in \calcprog\ is a lot
simpler and it doesn't seem to have bad convergence problems (of
course, it hasn't been extensively tested yet).  I've
taken a very ``first order'' approach.  Instead of damping the changes
in orbital occupations, I damp the changes in \Hii.  
So the orbital occupations are not played around with, i.e.:
\begin{equation}
p_i^{(k+1)}(initial) = p_i^{(k)}(final)
\end{equation}
Equation \ref{eqn:old_net_chg_def} is still used to generate the net
charges. 
Then, an undamped new value of \Hii\ is calculated:
\begin{equation}
\Hii^{'} = A_i (Q_{\nu}^{(k+1)})^2 + B_i Q_{\nu}^{(k+1)} + C_i  
\label{eqn:Hiinew_def}
\end{equation}
and \Hiip\ is
calculated as follows:
\begin{equation}
\Hiip = \Hiik + \lambda (\Hii^{'} - \Hiik)
\end{equation}


That's it.  It's very first order, but it seems to work reasonably
well.  It's worth testing this whole thing some more.


\end{document}
