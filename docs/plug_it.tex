%%%%%%%%%%%%%%%%%%%%%%%%%%%%%%%%%%%%%%%%%%%%%%%%%%%%%%%%%%%%%%%%%%%%

\chapter{What is \prog\ and why should I use it?}

\prog\ is a group of programs for performing extended \huek\
calculations \cite{eht1,eht2} and analyzing and visualizing the results. 
The programs \calcprog\ and \viewprog\ form the core of the package.

\section{\calcprog}

\calcprog\ is the program which performs the actual extended
\huek\ calculations.  It can be used to perform calculations on both
isolated molecules and extended systems of 1, 2, or 3 dimensions. 

\calcprog\ is an almost complete rewrite of the program \oldprog\
and is written almost entirely in C (the routines for
evaluating overlap matrix elements and diagonalizing the
Hamiltonian matrix from \oldprog\ remain)
Because of the fact that all memory
used in the program is allocated dynamically, there are no
restrictions on the number of atoms, K points, or orbitals which can
be used (this isn't totally true:  there is a limit of 20 user defined
atom types).  The only limitation is the amount of memory that your
computer has and the length of time which you are willing to wait for
the run to finish.

Because of the fact that \calcprog\ is written to be easy to
maintain and understand, we have not spent a lot of
time trying to make it fast.  This isn't to say that it's slow, but it
certainly could be faster.

The input files to \calcprog\ are keyword based, so, with a few
exceptions, it doesn't really matter what order things are in.  In
addition, white space (spaces, tabs, etc.) in the input are ignored.

Here are just a few reasons to use \calcprog:
\begin{itemize}

\item Built in parameters for most elements.

\item Gaussian style Z--matrix, standard Cartesian, or
crystallographic coordinate input. 

\item Automatic generation of points along a reaction coordinate (for
Walsh diagrams).

\item Particular pieces of information can be monitored at each step
along a reaction coordinate (e.g. the reduced overlap population
between two atoms can be printed at each step along a Walsh diagram).

\item Automagic generation of K points along symmetry lines for band
structures.

\item Orientation independent determination of symmetry elements.

\item More symmetry elements are found (up through S$_8$).

\item DOS and COOP data are in ASCII format, so you can plot the data
with any plotting program... though you will {\em want} to use
\viewprog\ :-).

\end{itemize}

%%%%%%%%%%%%%%%%%%%%%%%%%%%%%%%%%%%%%%%%%%%%%%%%%%%%%%%%%%%%%%%%%%%%

\section{\viewprog}

\viewprog\ is an X--Windows based, interactive program for displaying
and printing the results obtained using \calcprog\ (though there's no
reason that it can't be made to display data from other programs).

Some of \viewprog's features are:

\begin{itemize}

\item Interactive 3-D manipulation of molecular structures.

\item Support for extended systems: ``grow'' crystals of any size.

\item Postscript output.

\item It doesn't use Motif.

\item Ability to place as many structures and graphs as desired on the same page. 

\item Numerous options for displaying molecules and MO surfaces.

\item Automatic generation of input files for \genprog{rayshade}, a
freeware raytracing program.  This can be used to get extra gratuitous
color 3D plots.

\item The official Roald Hoffmann seal of approval on the way the output looks.
 ({\bf NOTE:} this feature is still under development.)

\end{itemize}

Both \calcprog\ and \viewprog\ were written to be as easy to port to
other flavors of UNIX as possible.  This is one of the
reasons why \viewprog\ doesn't use any of the snazzy user interface
libraries which are available.


