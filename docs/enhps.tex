%%%%%%%%%%%%%%%%
%%%%%%%%%%%%%%%%
%%%%%%%%%%%%%%%%
\chapter{Using the enhanced Postscript features in \viewprog}

It is now possible to include superscripts, subscripts, different
fonts, and all kinds of other wacky stuff in output from \viewprog.
This is accomplished using an adaptation of the enhpost terminal type
from gnuplot.  If you are familiar with this, then you don't have to
read this section, everything is exactly the same here {\em except}
that you can't do rotated text.

To get a superscript, use the \^{} symbol.  \^{} only holds for one
character unless a group of characters is enclosed in squiggly
brackets (\{ and \}), so the line:
\begin{verbatim}         X^10 \end{verbatim}
comes out looking like: 
\begin{quotation}
X$^10$,
\end{quotation}
while the line 
\begin{verbatim}         X^{10} \end{verbatim} 
gives:
\begin{quotation}
X$^{10}$. 
\end{quotation}
To get a subscript, use the \_ symbol.  \_ behaves in the same manner
as \^\ . 

To change font, put a backslashed version of the name of the font
inside squiggly brackets with the text to be changed.  For example
\begin{verbatim} look at {/Symbol G} now \end{verbatim}  
gives: 
\begin{quotation}
look at $\Gamma$ now. 
\end{quotation}

There are some other features of the enhpost drivers, but these are
the most important.
