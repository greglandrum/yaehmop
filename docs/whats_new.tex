%%%%%%%%%%%%%%%%%%%%%%%%%%%%%%%%%%%%%%%%%%%%%%%%%%%%%%%%%%%%%%%%%%%%
\chapter{What's new in version 3.0?}

Quite a few new features have been added to \prog\ and one or two
'problems' have gone away :-) 

\noindent The \prog\ development team has also increased in size as of
this release ... so please
use the version 3.0 citations when publishing
results generated with \prog. Thanks ! \\[0.1in]


\noindent Additions to \calcprog\ include ...

\begin{itemize}

\item f-orbitals ... at last !!!

\item COOP's in a fragment molecular orbital (FMO) basis

\item Hamilton population analysis: a tool for total energy partitioning
built on the orbital, atom and fmo COOP options offered in \prog \cite{cohp1,cohp2,cohp3}

\end{itemize}

\noindent also the 'known bugs' section of the 'those damn bugs' chapter of the version 2.0 manual has vanished ... so there are absolutely {\bf NO} bugs left in \prog\ :-) \\[0.1in]



%%%%%%%%%%%%%%%%%%%%%%%%%%%%%%%%%%%%%%%%%%%%%%%%%%%%%%%%%%%%%%%%%%%%

\chapter{What's new in version 2.0?}

A bunch of stuff has been added to this version.  This is almost
certainly the last non-bug fix release of the programs until I
graduate.

\begin{itemize}

\item Numerous bug fixes.

\item Much improved MO plotting.  Including Jorgenson and
Salem (or CACAO) style plots that can be rotated in ``real-time'' to
find the optimal viewing angle.  Also included are contour plots
of MOs.

\item Fragment Crystal Orbital analysis, a new interpretive tool
for crystalline systems.

\item A new keyword allowing diagonalization of the hamiltonian
without including the overlap matrix.

\item \viewprog\ now provides distances, angles, and dihedral angles
between selected atoms in molecules.

\item There are several new options for display of molecules in
\viewprog.  Including tube bonds and pseudo-3D crosses.

%\item Support for independently defined ``Geometry Fragments'' in
%\calcprog.  These can be used to define complicated pieces of a molecule or
%crystal or to break a geometry definition into both cartesian and
%Z-matrix input.


\end{itemize}

%%%%%%%%%%%%%%%%%%%%%%%%%%%%%%%%%%%%%%%%%%%%%%%%%%%%%%%%%%%%%%%%%%%%

\chapter{What's new in version 1.2?}

\begin{itemize}

\item Support for using LAPACK routines to diagonalize the matrices. 

\item A version for Power Macs.

\item I have more faith in the MO drawings now.  The normalization
constants were right and now that I evaluate them in atomic units the
pictures look right too.  Thanks Grisha!

\item More juicy raisins in every bite!

\item Other things that escape me at the moment.

\end{itemize}



