\chapter{\prog\ on the Macintosh}

As of version 1.2 of \prog, there is a Macintosh port of everything.
At the moment, the Mac version only runs on Power Macs.
A port to the 68K based Macs is not suported.

%There hasn't been much testing of \prog\ on Power Macs
%configured differently than what we have in the group, but it's a fairly
%safe bet that \viewprog\ will not work on displays that have
%less than 256 colors available.  There's also a distinct possibility
%that it won't work on displays with more than 256 colors.  To be safe,
%go to the Monitors control panel and switch to 256 color mode before
%running \viewprog.

The Mac port was done using the CodeWarrior compiler from Metrowerks.
Source code and project files for the Metrowerks IDE are available
upon request.
Codewarrior is fantastic ! Metroworks prices it reasonably, includes a ton of
useful examples and libraries, and has an
excellent upgrade policy.  In addition, the MW technical support is
{\bf excellent}.

The Fortran bits of the program were converted using f2c on our
workstations, and then compiled on the Mac using a port of the f2c
libraries.  All input and output that would normally go to the
console on a workstation is handled by the SIOUX library included with
CodeWarrior.  The basic structure of the graphics stuff used in
\viewprog\ was done using the EasyApp application shell distributed
with CW.

\section{A couple of disclaimers}
The Mac version of \prog\ is not the most beautiful thing that the
world has ever seen.  Some of the operations are handled in an ugly,
non-Mac way.  This is a direct consequence of the program's
Unix heritage.  Hopefully, in some future version these 
difficulties will be eliminated.

The Mac version of the programs are not nearly as stable as the UNIX
version, principally because the MacOS isn't a
protected mode operating system.

\section{Using \calcprog\ on a Macintosh}

When \calcprog\ starts up it will open a standard file choice dialog,
you should choose the input file in that dialog box.  If the program
has problems opening the parameter file (usually called {\tt
eht\_parms.dat}), it'll pop up another dialog box.  You should use that
dialog box to find and select the parameter file.  You can avoid this
by having a copy of the parameter file in the same folder as the input file.
To work around this make an alias for {\tt eht\_parms.dat}, copy it to 
the input folder, and then rename it {\tt eht\_parms.dat}.

%\section{Using \viewprog\ on a Macintosh}

%While the Mac version of \viewprog\ is very similar to the X version,
%there are a few differences:
%\begin{itemize}
%\item Instead of having button windows open up, new menus are added to the
%menu bar.   

%\item When opening files, you will not be prompted for the file
%name, but will be given a standard Mac file choice dialog.

%\item Filling of projected DOSs is not done on screen.
%If curve filling is turned on, the printed output will
%be filled properly.

%\item Line styles are indicated with color on screen.  This is because
%Quickdraw doesn't seem to support dashed line styles.  The Postscript
%files generated still use dashed line styles like on the Unix version.

%\item Breaking lines and tube bonds are not always drawn properly on screen.
%This is due to another problem with Quickdraw, which does not define
%line thicknesses relative to the center of the line.  Again, the
%Postscript output is fine.

%\item Standard mac printing isn't up and running yet, but you can still
% create a postscript file and print that just as you would print any
% postscript file from the mac.  
%This will probably be fixed in a future version.

%\item Because the Mac mouse only has a single button, some of the
%selection features work differently.  You must be in {\sf Select} mode
%to change active objects.

%\end{itemize}


\section{The fitting programs}

The fitting programs will open a file choice dialog on start up.  You
should pick the {\em input} file used to run the calculation.

\section{General Mac hints}

If you get errors about the programs not having enough memory or not
being able to allocate matrices, increase the size of the memory
allocation for the troublesome program.  If you don't know how to do
this:  select the application you want to change, select ``Get
Info'' from the File menu (or hit CMD-I), then increase the
``Preferred Size'' entry. 
%%%%%%%%%%%%%%%%%%
%%%%%%%%%%%%%%%%%%

