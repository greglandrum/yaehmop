%%%%%%%%%%%%%%%%%%
%%%%%%%%%%%%%%%%%%
\chapter{A word (or two) from Greg ... some acknowledgements}

The members of the Hoffmann group were invaluable in the development
and testing of these programs.  They provided moral support, bug
reports, featur suggestions, and esthetic criticisms that were
invaluable.  The group 
members most involved were: Hugh Genin,  Norman Goldberg, Kimberly
Lawler, Qiang Liu,  Erika Merschrod, Udo Radius, Grigoriy Vajenine
(who pointed out  
that I should be evaluating the radial parts of wavefunctions in
atomic units), and Kazunari
Yoshizawa.  In addition, the students and auditors of Chemistry 798 in
the fall of 1994 and the spring of 1995. acted as unwitting
beta-testers and uncovered a few   
problems I never would have found.  Roald Hoffmann (my advisor) was
very supportive of my efforts and never chastised me for how much
time I was devoting to this project.  Of course, he {\bf did} accuse
the program of being vaporware, but this should nip that criticism in
the bud.  Thanks Roald!   

Edgar M\"{u}ller has provided a number of helpful suggestions as well
as some Fortran code which served as the template for the
code to deal with crystallographic coordinates.  Edgar also came up
with a consistent parameter set for the entire periodic table.  This 
parameter set is distributed with this release of the program as {\tt
muller\_parms.dat}.

Paul K\"{o}gerler has also provided suggestions for features which are
now integrated and has even agreed to add some features himself.
Expect to see these in a future release.

The function used to diagonalize the inertia tensor as part of the
symmetry analysis is taken from the {\sf meschach} library.  This is a
freely available package of functions written in C for working with
matrices. {\sf meschach} was written by David Stewart and Zbigniew Leyk
at the Australian National University.  If you want to use this
function in your own code, please obtain a copy of the entire library.
I want to go ahead and 
take the chance to thank David Stewart for making this library freely
available.  At this point I can't help but interject a piece of 
propaganda: free software is top quality stuff, find out about it and
use it! 

The basis of the code to calculate solid isosurfaces was taken from
the article ``An Implicit Surface Polygonalizer'' by Jules Bloomenthal
in {\em Graphics Gems IV}, Academic Press, 1994.  If you do graphics,
taking a look at these books is a really good idea.  

The algorithm used to do hidden line removal in the Jorgenson and
Salem style MO plots is a slight modification of that used in
Jorgenson's {\sf PSI88} program.  Because {\sf PSI88} is written in
Fortran, none 
of the code from the program was used, I just used {\sf PSI88} to
figure out the algorithm.

The enhanced postscript code is adapted from the file enhpost.trm for
{\sf gnuplot} version 3.5.  The original code was written by David Denholm
and Matt Heffron, both of whom have given me permission to distribute
this adaptation. 

The code used to contour data (both for FCO and MO plots) is adapted
from that used in {\sf gnuplot} version 3.5 (I love borrowing code from
{\sf gnuplot}!).  The original code was written by Gershom Elber and this
modification is distributed with his permission.

