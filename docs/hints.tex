%%%%%%%%%%%%%%%%
%%%%%%%%%%%%%%%%
%%%%%%%%%%%%%%%%
\chapter{ Some (hopefully) helpful hints}

\section{Choosing how many overlaps to use}

The overlaps specified determine how many unit cells the program
uses when building the overlap matrix in K space.
The important thing when answering this question is to remember that
the goal is to include all unit cells surrounding the 'home' cell 
that have a non-zero contribution to the overlap
matrix. The general criterion here is that you should
go out far enough that the length of the lattice vector times the
number of overlaps is between 10 and 20 \AA.  Some systems don't
require this many overlaps, and some require more.  It's safe to go
with too many overlaps, though this causes \calcprog\ to use more
memory and go slower.  If you don't have enough overlaps, the
diagonalization procedure will fail.  This will be reported in the
status file after the program finishes running.


\section{Choosing the number of k points to use}

This is a tricky question.  The right answer is that you should always
do a k point convergence test for every calculation (i.e. you should
try using a variety of different sampling densities and stop when you
get convergence).   However, this isn't always practical or possible.
The general guideline we use is that the number of crystal orbitals
(number of orbitals in the unit cell times the number of k points)
should be equal to 1000.  This criterion is highly questionable when
doing slab models of interfaces or surfaces, so be careful with these
systems.

\section{Choosing the number of points in band structures}

Generally using 40 k points per symmetry line works fine.  If your
bands are flat, you can use less than this.  If you are really worried
about seeing weakly avoided crossings and you can't tell if you are
seeing one, use more points.


\section{Calculations on big systems}

When doing calculations on large systems (where the meaning of large
depends on how much memory your computer has), it's very good idea to
do the average properties and band structure calculations separately.
This is because average properties calculations use a lot more memory,
and band structure calculations use a lot more k points.  If you are
nearing the limit of the memory available on your machine because of
the demands of the average properties calculation, the band structure
calculation will take much much longer than it has too.  You are
better off if you do the two runs separately.  It's also a good idea
to do the band structure once, then comment out the band part of the
input file.  That way if you add projected DOS's or COOPs later, you
won't accidentally redo the band structure, which won't have changed. 

%\section{Using \viewprog}

%When moving or scaling graphs (property, band, walsh, whatever) in
%\viewprog, it's a good idea to turn off the curves.  Everything will
%update a hell of a lot faster if you do.  Once you have everything in
%place, turn the curves back on. 

%The 3D perspective code in \viewprog\ is stupid and primitive.  It
%works okay, but the interface is clunky and it is lacking some
%features that should be there.  The biggest of these is the lack of
%near plane clipping.  What this means to you, the user, is that
%objects behind the camera which is showing you the image will show up
%on screen, they'll just be inverted.  To work around this problem,
%either move the molecule farther away from you, or adjust the Z
%scaling. 

